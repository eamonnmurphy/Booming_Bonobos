\documentclass[10pt]{article}
\usepackage[utf8]{inputenc}
\usepackage{graphicx}
\usepackage{a4wide}
\usepackage{geometry}
\addtolength{\topmargin}{-1.5in}
\newgeometry{top = 0cm, bottom = 1cm}

\title{Is Florida getting warmer?}
\author{Booming Bonobos - Eamonn, Chalita, Emma, Lizzie and Uva}
\date{October 2021}

\begin{document}

\maketitle

\thispagestyle{empty}

\section{Introduction}
By analysing a dataset containing average yearly temperatures in Key West, 
Florida, we can deduce whether Florida is getting warmer over time. In order to 
do this, we can calculate the correlation coefficient between the temperatures 
and the years.

\section{Methods}
A script was created in R to analyse the dataset. Spearman's correlation 
coefficent was calculated between the years and the average temperatures. In 
order to calculate a p-value for the increase in temperatures, permutation testing
 was used. The order of temperatures was shuffled randomly between years 10,000 
 times, and the correlation coefficient calculated for each random order. The 
 p-value would thus be the number of random correlations greater than the base 
 correlation, divided by 10,000.

\section{Results}
\begin{center}
    \includegraphics[scale = 0.4]{../results/temp_year_scatter.png}
    \includegraphics[scale = 0.4]{../results/coeff_distro.png}

    Figure One: (A) Temperature (C) vs. Year for Florida dataset. Line of best fit
    added with confidence interval. (B) Distribution of correlation coefficients
    for randomly permuted orders.
\end{center}

The correlation coefficent between the temperature and the year came to 0.526.
10,000 permutation tests were run, randomly shuffling the years. The Spearman's
correlation coefficent was calculated for each of these random orders. The
distribution of these is displayed in Figure One (B). None of these correlations 
were greater than the originally calculated correlation, meaning that the p-value 
is less than 1/10000 (<0.0001). This indicates that the observed increase in 
temperature over the century of data is very likely to be a true increase, not 
due to random chance. This is likely to be an effect of climate change, although
it could be a more regional weather pattern also. Further analysis would be needed 
to determine this.

\end{document}